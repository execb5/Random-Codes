\documentclass[12pt]{article}
\usepackage[brazilian]{babel}
\usepackage[utf8]{inputenc}
\usepackage[T1]{fontenc}

\sloppy

\title{Organização e Arquitetura de Computadores III\\ Resumo do Artigo: Cache Profiling and the SPEC Benchmarks: A Case Study}

\author{Matthias Oliveira de Nunes}

\begin{document}

\maketitle

\section{Resumo}

Cache só funciona bem para programas que demonstram uma localidade suficiente,
outros programas, que possuem padrões de referênciação diferentes, não conseguem
utilizar tão bem a cache. As memórias cache funcionam porque a maioria dos
programas demonstram uma localidade significativa. Localidade temporal existe
quando um programa referencia a mesma localização de memória muitas vezes em um
período curto de tempo.

Existem propriedades da memória Cache que programadores podem tirar proveito, já
que um programa pode perder muito tempo de execução buscando por algo que não
está na cache. Caches têm três parâmetros que as definem: Capacidade, tamanho de
bloco e associatividade.

Sabendo isso sobre a Cache, existem tecnicas que podem ser aplicadas para
melhorar o desempenho de um programa.

\subsection{Mesclar vetores}

Quando um programa referencia dois ou mais vetores da mesma dimensão usando o
mesmo índice. Mesclando esses vetores em um único vetor, isso aumenta a
localidade espacial e reduz potencialmente o número de miss.

\subsection{Preenchimento e alinhamento de estruturas}

Preencher uma estrutura de dado até um múltiplo do tamanho do bloco e alinhando
eles em um limite de bloco pode eliminar miss por desalinhamento, que geralmente
geram miss de conflito.

\subsection{Empacotamento}

Empacotamento é o oposto do preenchimento. Reduzindo um vetor para o menor
espaço possível, o programador vai aumentar a localidade espacial e conseguir
reduzir o número de miss de capacidade e conflitos.

\subsection{Fusão de laços}

Como o nome mesmo diz, é ter varios laços que trabalhem sobre uma estrutura de
dado, como um vetor, e combinar esses laços em um único laço. Isso aumenta a
localidade e o número de miss de capacidade irá reduzir.

\subsection{Bloqueamento}

É uma técnica que estrutura um programa para re-usar pedaços de dados que cabem
dentro da memória Cache para reduzir miss de capacidade.

\end{document}
